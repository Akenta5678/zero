\documentclass[reqno]{article}
\usepackage[utf8]{inputenc}
\usepackage{lmodern}
\usepackage{blindtext}
\usepackage{amsmath}
\usepackage{relsize} %bigger math symbols
\usepackage{bm} %bolder text
\usepackage{ulem} %Fixes underlining line break issue
\usepackage[a4paper, inner=1.7cm, outer=2.7cm, top=3cm, bottom=3cm, bindingoffset=1.2cm]{geometry}
\usepackage{tcolorbox}
\usepackage{graphicx}
\usepackage{wrapfig}
\usepackage{enumitem}
\newlist{terms}{description}{1}
\begin{document}
\title{\textbf{ICR Equality is Maintained in Batog-based Redistribution with Corrected Stakes}}
\date{September 2020}
\maketitle

\subsection*{\textbf{Overview of Proof}}


\begin{itemize}
  \item Background
  \item Problem - with uncorrected stakes, rewards can break CDP ordering
  \item Proposed solution: corrected stake
  \item Terminology
\end{itemize}

\bigskip
\textbf{Core Proofs}

\begin{itemize}
  \item Proof 1: $ICR_1=ICR_2$ in simple case. $1^{st}$ order, 1 past liquidation. Evolves to $2^{nd}$ order: 1 new stake, 1 subsequent liquidation
  \item Proof 2: $ICR_1=ICR_2$ in general case. $1^{st}$ order, M past liquidations. Evolves to $2^{nd}$ order: 1 new stake, 1 subsequent liquidation
\end{itemize}

\bigskip
\textbf{Extensions}

\begin{itemize}
  \item Proof 3: $ICR_1=ICR_2$ for $1^{st}$ order, M past liquidations Evolves to $2^{nd}$ order: 1 new stake, P subsequent liquidations
  \item Proof 4: $ICR_1=ICR_2$ for $1^{st}$ order, M past liquidations Evolves to $2^{nd}$ order: Q new stakes, P subsequent liquidations
  \item Show $2^{nd}$ order system is equivalent to first order 
  \item Show that $n^{th}$ order system is equivalent to first order
\end{itemize}

\bigskip
\subsection*{\textbf{Background}}

Previously, we showed that rewards proportional to collateral in a system of CDPs ordered by ICR preserves ordering across reward events. This assumed that rewards are proportional to the total collateral of the CDP.\\

In reality, the redistributions are implemented with a Batog pull-based mechanism, for computational efficiency.  In the Batog implementation, collateral and debt rewards are not compounded - they are stored separately from the CDPs initial collateral and debt, and are not included in future reward computations.  Each earned reward is based \textit{only} on the CDPs initial collateral “stake”.\\

However, the ICR of a CDP is always computed as the ratio of it’s total collateral to its total debt. That is, the terms in a CDP’s ICR calculation \textbf{do} include all its previous accumulated rewards.

\bigskip
\subsection*{\textbf{The Problem: Rewards Can Break CDP Ordering}}

As the system undergoes reward events, a given CDP’s ratio of initial collateral to its total collateral shrinks. Rewards are based on a smaller and smaller share of the total collateral. This is fine, as long as all active CDPs have experienced all reward events - in this case, ordering is maintained.\\

However, a problem arises when a new CDP is created after active CDPs have received reward shares.  This “fresh” CDP has then experienced fewer rewards than the earlier CDPs, and thus, it receives a disproportionate share of subsequent rewards, relative to its total collateral.\\

This means that across a reward event, a ‘fresh’ CDP’s \textit{proportional change} in ICR is different from the proportional change of the ICR of an older CDP, which has been in the system from the start.\\

This discrepancy can break CDP ordering.\\

\bigskip
\textbf{System Order Terminology}

We introduce the notion of \textit{system order}. In general, a system of CDPs increases from order N to order N+1 when the following sequence of events occurs:

\begin{itemize}
  \item 1 or more new CDPs are created 
  \item 1 or more CDPs are subsequently liquidated
\end{itemize}

We capture this in a system evolution function:

\begin{equation} 
    f(S_N)=S_{N+1}
\end{equation}

\bigskip
Let $S_1$ define a system of CDPs with past liquidations, in which all active CDPs have received reward shares from all past liquidations. $S_1$ is a first-order system, and contains only first-order stakes. Each stake $s_i$ is equal to it’s collateral $c_i$, and \textit{totalStakes} $= \sum s_i = \sum c_i$.\\

Let $S_2$ define an evolution of S1, i.e. $S_2 = f(S_1)$. $S_2$ is a system with past liquidations, with \textit{totalStakes} $= \sum s_i + \sum s_j$, where $s_j$ is the stake of newly added $CDP_j$. $S_2$ is a second-order system, containing \textbf{both} \textit{first-order} stakes $s_i = c_i$ which have experienced all liquidations, \textbf{and} \textit{second-order} stakes $s_j$ which have only experienced the liquidations after their creation.\\

In general, $S_n$ is a system with n sequential pairs, each consisting of a CDP creation period and a liquidation period. CDP's made in a given CDP creation period t have experienced only those liquidations that occurred in liquidation period t or greater.

\bigskip
\subsection*{\textbf{Corrected Stake Approach}}

To correct for the advantage gained by later stakes over earlier stakes, we introduce a corrected stake:

\begin{equation}
    s_i=
        \begin{cases} 
            c_i & for \;totalCollateral_\emptyset = 0\\
            \frac{\mathlarger{c_i} \cdot \; totalStakes_\emptyset} {\;totalCollateral_\emptyset} & for \;totalCollateral_\emptyset>0
        \end{cases}
\end{equation}

\bigskip
Where $totalStakes_\emptyset$ and $totalCollateral_\emptyset$ are the respective snapshots of the total stakes and total collateral in the system, taken immediately after the last liquidation event. Note that with this terminology, the total collateral includes the total stakes, and therefore $totalCollateral_\emptyset \ge totalStakes_\emptyset$

\bigskip
\textbf{At First-Order, Stake Equals Initial Collateral}

For first-order systems, all CDPs were added before any liquidation events occurred. The snapshot $totalCollateral_\emptyset$ is equal to 0. Therefore:

\begin{equation} 
    s_i=c_i
\end{equation}
for all $s_i$, $c_i$ in an $S_1$ system.

\bigskip
\textbf{Intuition Behind Choice of Corrected Stake, $s_i$}

The corrected stake $s_i$ is chosen such that it earns rewards from liquidations equivalent to a CDP that would have accumulated $c_i$ total collateral by the time the fresh $CDP_i$ was created.\\

The corrected stake effectively models the fresh CDP’s collateral $c_i$ as a total collateral, which includes ‘virtual’ accumulated rewards. The corrected stake earns rewards for the CDP as if the CDP was first-order, and had been in the system from the beginning - thus maintaining proportional reward growth.\\

We now prove that ICR equality is maintained with rewards proportional to corrected stakes - starting with the simplest case, and progressively generalizing.

\bigskip
\subsection*{\textbf{PROOF 1. Corrected Stake Preserves ICR Equality Across a Reward Event in a Second-Order System}}

We consider the following event sequence:

\begin{itemize}
  \item $n+1$ CDPs are created
  \item 1 CDP is liquidated
  \item 1 fresh CDP is created
  \item 1 old CDP is liquidated
\end{itemize}

\bigskip
In other words, a first-order system of $n+1$ CDPs undergoes 1 liquidation, and then evolves to second-order.\\

Each initial $CDP_i$ has collateral $c_i$ and debt $d_i$.\\

Then, the first liquidation occurs: $CDP_j$ with collateral $c_j$, debt $d_j$, is redistributed. The total number of active CDPs reduces from $n+1$ to $n$.\\

Each remaining active $CDP_i$ earns a collateral reward $x_i$, which is proportional to the liquidated collateral $c_j$, and $CDP_i$’s stake as a share of total stakes:

\begin{equation} 
    x_i=\frac{c_j \cdot s_i}{\sum\limits^{n+1}_{\substack{i=1\\i \ne j}}s_i}
\end{equation}

\bigskip
By equation 3), the property of a first-order system, $s_i$ equals $c_i$:

\begin{equation}
    x_i=\frac{c_j \cdot c_i}{\sum\limits^{n+1}_{\substack{i=1\\i \ne j}}c_i}
\end{equation}

\bigskip
For simplicity, let:

\begin{equation} 
    C_n=\sum^{n+1}_{\substack{i=1\\i \ne j}}(c_i)
\end{equation}

\bigskip
There are now $n$ CDPs in the system.\\

Let the sum of accumulated rewards over all active CDPs be given by: 

\begin{equation} 
    X_n=\sum^{n+1}_{\substack{i=1\\i \ne j}}x_i
\end{equation}

\bigskip
Note that after the liquidation, the system snapshots update:

\begin{equation} 
    totalStakes_\emptyset=C_n
\end{equation}

\begin{equation} 
    totalCollateral_\emptyset=C_n+X_n
\end{equation}

\bigskip
(Note also that $X_n=c_j$ and therefore $totalCollateral_\emptyset=C_n+c_j=C_{n+1}$)

\bigskip
Now, a fresh CDP is added, $CDP_F$, with collateral $c_F$.  Let the ICR of $CDP_F$ equal the ICR of an active first-order $CDP_G$:

\begin{equation} 
    ICR_F=ICR_G
\end{equation}

\begin{equation} 
    ICR_F=\frac{c_\mathsmaller{F}}{d_\mathsmaller{F}}
\end{equation}

\begin{equation} 
    ICR_G=\frac{c_\mathsmaller{G}+x_\mathsmaller{G}}{d_\mathsmaller{G}+y_\mathsmaller{G}}
\end{equation}

\bigskip
Where $c_F$, $d_F$ and $c_G$, $d_G$ are the collateral and debt values of $CDP_F$ and $CDP_G$ respectively.\\

$x_G$, $y_G$ are the respective accumulated collateral and debt rewards for $CDP_G$ earned by its stake over its lifetime.\\

The ICR equality identity 10) yields the following relation:

\begin{equation} 
        c_F=\frac{d_\mathsmaller{F}}{d_\mathsmaller{G}+y_\mathsmaller{G}}(c_\mathsmaller{G}+x_\mathsmaller{G})
\end{equation}

\bigskip
i.e.

\begin{equation} 
    c_F=k(c_\mathsmaller{G}+x_\mathsmaller{G})
\end{equation}

\bigskip
where

\begin{equation} 
    k=\frac{d_\mathsmaller{F}}{d_\mathsmaller{G}+y_\mathsmaller{G}}
\end{equation}

\bigskip
$CDP_F$’s stake $s_F$ is given by the corrected stake rule 2), that is:

\begin{equation} 
    s_F=\frac{c_\mathsmaller{F} \cdot totalStakes_\emptyset}{totalColateral_\emptyset}
\end{equation}

\bigskip
Which by 8) and 9) gives:

\begin{equation} 
    s_F=\frac{c_F \cdot C_n}{C_n+X_n}
\end{equation}

\bigskip
Now, a new liquidation occurs: $CDP_Z$ liquidates. The system becomes second-order.\\

The event causes $CDP_Z$’s collateral and debt ($c_Z$ and $d_Z$) to be redistributed between all active CDPs, proportional to their stake.\\

For simplicity, let :

\begin{equation} 
    a=\frac{c_Z+x_Z}{totalStakes}
\end{equation}

\begin{equation} 
    b=\frac{d_Z+y_Z}{totalStakes}
\end{equation}

\bigskip
We define the collateral and debt rewards earned by $CDP_F$ and $CDP_G$ in the reward event:

\begin{equation}
    \begin{split}
        r_{c\mathsmaller{F}}=as_\mathsmaller{F}\\
        r_{d\mathsmaller{F}}=bs_\mathsmaller{F}\\
        r_{c\mathsmaller{G}}=as_\mathsmaller{G}\\
        r_{d\mathsmaller{G}}=bs_\mathsmaller{G}
    \end{split}
\end{equation}

\bigskip
And since $s_\mathsmaller{G}$ is a first-order stake:

\begin{equation} 
    s_\mathsmaller{G}=c_\mathsmaller{G}
\end{equation}

\bigskip
To show ICR equivalence after the reward event, we must first obtain $s_F$ as a linear function of $c_G$. Recall our definition of $CDP_F$’s stake from 17):

\begin{equation} 
    s_\mathsmaller{F}=\frac{c_\mathsmaller{F} \cdot C_n}{(C_n+X_n)}
\end{equation}

\bigskip
Now, substituting in the expression for F’s collateral, 14), we obtain:

\begin{equation} 
    s_\mathsmaller{F}=\frac{k(c_\mathsmaller{G}+x_\mathsmaller{G})C_n}{C_n+X_n}
\end{equation}

\bigskip
Consider the term, $x_\mathsmaller{G}$, which represents the total accumulated past reward of $CDP_G$ before liquidation of $CDP_Z$.  Since $CDP_j$ was the only previous liquidation:

\begin{equation} 
    x_\mathsmaller{G}=\frac{s_\mathsmaller{G}c_j}{\sum\limits^{n+1}_{\substack{i=1\\i \ne j}}s_i}
\end{equation}

\bigskip
And since all stakes $s_i$ were first order:

\begin{equation} 
    x_\mathsmaller{G}=\frac{c_\mathsmaller{G}c_j}{C_n}
\end{equation}

\bigskip
By the same token, the total accumulated rewards for all active CDPs, from liquidation of $CDP_j$ must be:

\begin{equation} 
    X_n=c_j
\end{equation}

\bigskip
Substituting these expressions for $x_\mathsmaller{G}$ and $X_n$ into 23), yields:

\begin{equation} 
        s_\mathsmaller{F}=\frac{k(c_\mathsmaller{G}C_n+c_\mathsmaller{G}c_j)}{C_n+c_j}
\end{equation}

\bigskip
Factoring out $c_G$ from the numerator:

\begin{equation} 
    s_\mathsmaller{F}=\frac{k c_\mathsmaller{G}(C_n+c_j)}{(C_n+c_j)}
\end{equation}

\bigskip
And cancelling, we obtain:

\begin{equation} 
    s_\mathsmaller{F}=kc_\mathsmaller{G}
\end{equation}

\bigskip
We now compare ICRs of $CDP_F$ and $CDP_G$, after liquidation of $CDP_Z$.

\begin{equation} 
    ICR_{F \; After}=\frac{c_\mathsmaller{F}+r_{c\mathsmaller{F}}}{d_\mathsmaller{F}+r_{d\mathsmaller{F}}}
\end{equation}

\begin{equation} 
    ICR_{G \; After}=\frac{c_\mathsmaller{G}+x_\mathsmaller{G}+r_{c\mathsmaller{G}}}{d_\mathsmaller{G}+y_\mathsmaller{G}+r_{d\mathsmaller{G}}}
\end{equation}

\bigskip
Using 20), the individual rewards as functions of stakes:

\begin{equation} 
    ICR_{F \; After}=\frac{c_F+as_F}{d_F+bs_F}
\end{equation}

\begin{equation} 
    ICR_{G \; After}=\frac{c_\mathsmaller{G}+x_\mathsmaller{G}+as_\mathsmaller{G}}{d_\mathsmaller{G}+y_\mathsmaller{G}+bs_\mathsmaller{G}}
\end{equation}

\bigskip
Now, substituting our definitions for $s_\mathsmaller{G}$ (21) and $s_F$ (29):

\begin{equation} 
    ICR_{F \; After}=\frac{c_\mathsmaller{F}+akc_\mathsmaller{G}}{d_\mathsmaller{F}+bkc_\mathsmaller{G}}
\end{equation}

\begin{equation} 
    ICR_{G \; After}=\frac{c_\mathsmaller{G}+x_\mathsmaller{G}+ac_\mathsmaller{G}}{d_\mathsmaller{G}+y_\mathsmaller{G}+bc_\mathsmaller{G}}
\end{equation}

\bigskip
Using identities 14) for $c_F$, and 15) for $d_F$:

\begin{equation} 
    ICR_{F \; After}=\frac{k(c_\mathsmaller{G} + x_\mathsmaller{G}+ac_\mathsmaller{G})}{k(d_\mathsmaller{G}+y_\mathsmaller{G}+bc_\mathsmaller{G})}
\end{equation}

\begin{equation} 
    ICR_{G \; After}=\frac{c_\mathsmaller{G}+x_\mathsmaller{G}+ac_\mathsmaller{G}}{d_\mathsmaller{G}+y_\mathsmaller{G}+bc_\mathsmaller{G}}
\end{equation}

\bigskip
Cancelling $k$:

\begin{equation} 
        ICR_{F \; After}=\frac{c_\mathsmaller{G} + x_\mathsmaller{G}+ac_\mathsmaller{G}}{d_\mathsmaller{G}+y_\mathsmaller{G}+bc_\mathsmaller{G}}
\end{equation}

\begin{equation} 
    ICR_{G \; After}=\frac{c_\mathsmaller{G}+x_\mathsmaller{G}+ac_\mathsmaller{G}}{d_\mathsmaller{G}+y_\mathsmaller{G}+bc_\mathsmaller{G}}
\end{equation}

\bigskip
Thus:

\begin{equation} 
    ICR_{F \; After}=ICR_{G \; After}
\end{equation}

\bigskip
QED.

\subsection*{PROOF 2. Corrected Stake Preserves ICR Equality Across a Reward Event in a Second Order System with $m$ Past Liquidations}

We now extend the above proof to cover the following event sequence:

\begin{itemize}
  \item $n+m$ CDPs are created
  \item $m$ CDPs are liquidated
  \item A fresh CDP is created
  \item An old CDP is liquidated
\end{itemize}

\bigskip
In other words, a first-order system of $n+m$ CDPs undergoes $m$ CDP liquidations, before evolving to second-order.  All other conditions remain the same.\\

Consider the $m$ past liquidations from the point of view of an active first-order $CDP_i$. As per 2), the stake of $CDP_i$ is $s_i = c_i$.\\

$CDP_i$ earns total accumulated reward, $x_i$  the sum of its rewards from $m$ liquidations.\\

With each liquidation, $c_j$ collateral is removed from the system. Again, as per 2), stake equals collateral. Thus, the $totalStakes$ numerator in each liquidation is reduced by $c_j$, where $j$ denotes the index of the liquidated CDP.\\ 

(Let’s assume $n+1 ... n+m$ are the liquidated CDPs and $1 ... n$ are the ones staying, for simplicity)\\

Let

\begin{equation} 
    C_{n+m}=\sum\limits^{n+m}_{i=1}c_i
\end{equation}

\bigskip
and

\begin{equation} 
    L_m=\sum\limits^m_{j=1}c_{n+j}
\end{equation}

\bigskip
We now sum all reward events, noting that the liquidated CDP’s collateral is removed from the $totalStakes$ numerator at each reward:

\begin{equation} 
    x_i=c_i\left[\frac{c_{n+1}+x_{n+1}}{C_{n+m}-L_1}+\frac{c_{n+2}+x_{n+2}}{C_{n+m}-L_2}+\frac{c_{n+3}+x_{n+3}}{C_{n+m}-L_3}+...+\frac{c_{n+m}+x_{n+m}}{C_{n+m}-L_m}\right]
\end{equation}

\bigskip
i.e.

\begin{equation} 
    x_i=c_i\sum\limits^m_{j=1}\frac{c_{n+j}+x_{n+j}}{\sum\limits^{n+m}_{i=1}c_i-\sum\limits^j_{p=1}c_{n+p}}
\end{equation}

\bigskip
(Note, that for liquidation of a given $CDP_j$, the redistributed collateral is the sum of its collateral $c_{n+j}$ plus it’s accumulated collateral reward $x_{n+j}$ which has itself been earned from liquidations $[n+1, n+2, n+3, … n+j-1]$.  Thus, liquidations have a “roll-up” effect - though, it is not important for our result. Actually, it can also be proved that $x_i=c_i\frac{L_m}{C_n}$)\\

We label the main sum expression $H$.\\

Rewriting $CDP_i$’s accumulated reward:

\begin{equation} 
    x_i=Hc_i
\end{equation}

\bigskip
Summing over all $n$ active CDPs gives the total accumulated rewards for active CDPs in the system:

\begin{equation} 
    X_n=\sum\limits^n_{i=1}Hc_i
\end{equation}

\begin{equation} 
    X_n=H \; C_n
\end{equation}

\bigskip
(Note: It can also be proved that $X_n=L_m$)

\bigskip
Now, a fresh CDP is added, $CDP_F$, with collateral $c_F$.  Let the ICR of $CDP_F$ equal the ICR of an active first-order $CDP_G$.\\

Now, $CDP_Z$ liquidates. Upon liquidation, the second-order $CDP_F$ and the first-order $CDP_G$ earn the following collateral rewards:

\begin{equation}
    \begin{split}
        r_{cF}=as_F\\
        r_{dF}=bs_F\\
        r_{cG}=as_G\\
        r_{dG}=bs_G
    \end{split}
\end{equation}

\bigskip
where

\begin{equation} 
    a=\frac{c_Z+x_Z}{totalStakes}
\end{equation}

\begin{equation} 
    b=\frac{d_Z+y_Z}{totalStakes}
\end{equation}

\bigskip
And since $s_G$ is a first-order stake:

\begin{equation} 
    s_G=c_G
\end{equation}

\bigskip
Again, we now seek $s_F$ as a linear function of $c_G$.\\

The stake $s_\mathsmaller{F}$ can be derived in exactly the same manner as in the simple case. By our corrected stakes rule, and the fact that $ICR_F = ICR_G$, we obtain:

\begin{equation} 
    s_\mathsmaller{F}=\frac{k(c_G+x_G)C_n}{C_n+X_n}
\end{equation}

\begin{center}
   \textit{(see steps 8-23 for this derivation)} 
\end{center} 

\bigskip
Substituting in the expressions for accumulated reward $x_i$ from 45), and total accumulated reward $X_n$ from 47):

\begin{equation} 
    s_\mathsmaller{F}=\frac{k(c_G+Hc_G)C_n}{C_n+HC_n}
\end{equation}

\bigskip
And factorizing:

\begin{equation} 
    s_F=\frac{kc_G(C_n+HC_n)}{(C_n+HC_n)}
\end{equation}

\bigskip
Canceling yields:

\begin{equation} 
    s_F=kc_G
\end{equation}

\bigskip
We obtain the same result for $s_F$ as in the single liquidation case 29) .Comparing ICRs as per 30) and by following same steps thereafter, yields:

\begin{equation} 
    ICR_{F \; After}=ICR_{G \; After}
\end{equation}

\bigskip
\subsection*{EXTENSION PROOF 3. Arbitrary Number of Liquidation Events At Current System Order}

If instead of a single liquidation event at a given system order, we have $P$ liquidation events, it is clear that ICR equality holds across all $P$ events:\\

Since ICR equality holds across one liquidation event, it will hold across the next, and thus hold for all.\\

Liquidation events do not alter the stakes that earn shares of liquidated collateral and debt - and for a given stake, the individual CDP reward term $x_i$ given in 4) depends only on reward sizes and the total stakes.

\bigskip
\subsection*{EXTENSION PROOF 4. Arbitrary Number of CDPs Added Between Liquidation Events}

With $N$ second-order CDPs added between consecutive liquidation events, the stake $s_F$ of any given second-order CDP is given by 1): 

\begin{equation} 
    s_F=\frac{c_F \cdot totalStakes_\emptyset}{totalColateral_\emptyset}
\end{equation}

\bigskip
The snapshots of the system state after the last liquidation event ($totalStakes_\emptyset$, $totalColateral_\emptyset$) remain constant until the next liquidation. It is clear that all $N$ second-order stakes $s_F$ have been corrected by the same constant factor.\\

Thus, $s_F$ in the N second-order CDPs case is equal to $s_F$ in the single second-order CDP case.\\

As such, the logic of Proof 2 applies - and ICR equality between a second-order CDP and first-order CDP holds across a liquidation event, no matter how many fresh CDPs are added in between.

\bigskip
\textbf{CONCLUSION 1}

\bigskip
Combining Proof 2 with Extensions 1 \& 2 yields the following conclusion:\\

\uline{In a second order system with $M$ previous liquidations, and $N$ second-order CDPs added after the last liquidation, ICR equality between a first-order CDP and second-order CDP holds across $P$ subsequent liquidation events.}\\

\textbf{2nd Order Systems Collapse to 1st Order}\\
We now show that a second-order system is equivalent to a first-order system.\\

Consider a hypothetical first order $CDP_1$ and an actual second order $CDP_2$. Let both CDPs have identical ICR, and also let $CDP_1$’s total collateral and debt equal $CDP_2$’s initial collateral and initial debt respectively:

\begin{equation} 
    c_1+x_1=c_2
\end{equation}

\begin{equation} 
    d_1+y_1=d_2
\end{equation}

\bigskip
Clearly, the ratio  $k = \frac{d_2}{(d_1+y_1)} = 1$.\\

We substitute $k=1$ into the second-order system expression for $s_F$, from equation 55), to yield:

\begin{equation} 
    s_2=c_1
\end{equation}

\bigskip
Thus, any second-order stake is equivalent to some hypothetical first-order stake $s_1=c_1$, which has accumulated collateral reward $x_1=(c_2-c_1)$ and debt reward $y_1=(d_2-d_1)$.\\

Therefore any second order system is equivalent to a first order system that contains only first-order stakes which have experienced all liquidations. We write:

\begin{equation} 
    S_2=S_1
\end{equation}

\bigskip
\textbf{N’th Order Systems Collapse to 1st Order}\\
Recall our system evolution function: 

\begin{equation} 
    f(S_N)=S_{N+1}
\end{equation}

\bigskip
By definition, our $N$’th order system is the $N$’th evolution of a first order system:

\begin{equation} 
    S_N=f^N(S_1)
\end{equation}

\bigskip
And thus the $N-1^{th}$ evolution of a second-order system:

\begin{equation} 
    S_N=f^{N-1}(S_2)
\end{equation}

\bigskip
By equivalence 61):

\begin{equation} 
    S_N=f^{N-1}(S_1)
\end{equation}

\bigskip
Repeating the steps, it is clear that\\

$S_N=f^{N-2}(S_1)$

$S_N=f^{N-3}(S_1)$

...etc, \\

and finally:

\begin{equation}
    \begin{split}
        S_N=f(S_1)\\
        S_N=S_2\\
        S_N=S_1
    \end{split}
\end{equation}

\bigskip
Having shown all nth order systems are equivalent to a first order system, we now extend our previous conclusion to nth order systems:

\bigskip
\textbf{CONCLUSION 2}

\uline{In an $n^{th}$ order system with $M$ previous liquidations, and $N$ second-order CDPs added after the last liquidation, ICR equality between an $(n-1)^{th}$ order CDP and $n^{th}$ order CDP holds across $P$ liquidation events.}


\end{document}

