\documentclass[reqno]{article}
\usepackage[utf8]{inputenc}
\usepackage{lmodern}
\usepackage{blindtext}
\usepackage{amsmath}
\usepackage{relsize} %bigger math symbols
\usepackage{bm} %bolder text
\usepackage{ulem} %Fixes underlining line break issue
\usepackage[a4paper, inner=1.7cm, outer=2.7cm, top=3cm, bottom=3cm, bindingoffset=1.2cm]{geometry}
\usepackage{tcolorbox}
\usepackage{graphicx}
\usepackage{wrapfig}
\usepackage{enumitem}
\newlist{terms}{description}{1}
\begin{document}
\title{\textbf{ICR Equality is Maintained in Batog-based Redistribution with Corrected Stakes}}
\date{September 2020}
\maketitle

\tableofcontents

\section{Introduction}


The Liquity protocol $\cite{Whitepaper}$ issues a USD-pegged stablecoin LUSD that is redeemable at face value against ETH: any owner of LUSD can redeem their stablecoins for equivalent value in the underlying collateral at any time. The ETH paid to the redeemer is taken from the borrowers' collateralized debt positions ("troves") in ascending order of collateral ratio. In other words, the system uses the redeemed LUSD to repay the debt on the riskiest trove with the currently lowest individual collateral ratio, and transfers a corresponding amount of ETH from the trove to the redeemer. If the redeemed LUSD is larger than the debt on the riskiest trove, the system proceeds with the second riskiest trove, and so on. \\

To allow for efficient redemptions despite Ethereum's gas constraints, the system keeps troves ordered by ICR, so that it can iterate over the linked list starting from the bottom. It is not feasible to sort the list after every operation. \\

Given that Liquity's fallback liquidation mechanism redistributes the collateral and debt of a liquidated trove between all remaining active troves, it must be  ensured that redistributions do not break trove ordering. \\

In principle, redistribution in proportion to the collateral size of active troves maintains ordering, as can be easily shown. However, in practice, it is clear that redistributing in a "push" based manner - iterating over all troves and updating their collateral and debt - does not scale, and has computational complexity of $O(n)$. \\

Previous work by Batog et al $\cite{Batog}$ derived a scalable $O(1)$ method to assign proportional rewards, as long as the basis for the rewards ("stakes") do not change over time. The method is "pull-based": instead of adjusting all recipient positions upon every reward event, the update is deferred to the moment at which an owner changes their position. In the described approach rewards are stored separately from the initial stakes, and do not compound. That is to say, past accumulated rewards are not included in future reward calculations. \\

It turns out that this approach thus cannot be applied to Liquity as is. As the system undergoes reward events, a given trove’s ratio of initial collateral to its total collateral shrinks. Rewards are based on a smaller and smaller share of the total collateral. This is fine, as long as all active troves have experienced all reward events - in this case, ordering is maintained since all troves are affected by the same change. \\

However, a problem arises when a new trove is created after active troves have received reward shares.  Such "fresh" troves (with no accumulated rewards) would thus gain an advantage in redistributions over older troves whose stakes may be smaller than their actual collateral due to the liquidations that have taken place in the meantime. In other words, a “fresh” trove that has experienced fewer rewards than the earlier troves would receive a disproportionate share of subsequent rewards relative to its collateral. Though, a trove's collateral ratio must always be based on its entire collateral, which does include accumulated rewards.  \\

This discrepancy means that the reward distribution scheme described in $\cite{Batog}$ can break the ordering of troves by collateral ratio. To remedy this, we modify the original approach by introducing a "corrected stake", to ensure fresh troves do not receive disproportionate rewards. We then show that this corrected stake preserves trove ordering. \\


\section{Scalable Reward Distribution with fixed stakes}
A naive push-based implementation ("strawman approach") would iterate over all participants and compute the reward for each recipient $j$ separately whenever the system distributes rewards. Thus, at a reward event $t$, every recipient would receive the following reward share:

\begin{equation} 
    r_{j,t}=s_j \cdot \frac{R_t}{S_t}
\end{equation}

where $R_t$ is the reward distributed at $t$, $r_{j,t}$ the reward share of recipient $j$, $s_j$ the stake of recipient $j$, and $S_t$ the sum of the stakes of all recipients, i.e. $\sum s_i$.

Note that $s_j$ does not depend on $t$, and is thus fixed throughout multiple reward events.  \\

Based on this prerequisite, $\cite{Batog}$ suggests a scalable $O(1)$ method of distributing such rewards by deferring the reward computation. The total reward share of $j$ from all reward events that occur between time $t_1$ and $t_2$ can be written as a sum of its reward shares with the (fixed) stake $s_j$ being factored out:

\begin{equation}
    \sum\limits_{t=t_1+1}^{t_2} r_{j,t} = s_j \cdot \sum\limits_{t=t_1+1}^{t_2}\frac{R_t}{S_t}
\end{equation}

Let $Q_t$ denote the sum of all rewards per total staked amount up to instant $t$:

\begin{equation}
   Q_t = \sum\limits_{k=0}^{t}\frac{R_t}{S_t}
\end{equation}

Assuming stake $s_j$ is deposited at moment $t_1$ and then
withdrawn at moment $t_2 > t_1$, we can use the sum $Q_t$ to
compute the total reward share for participant $j$ since


\begin{equation} 
    r_j = s_j \cdot \sum\limits_{t=t_1+1}^{t_2}\frac{R_t}{S_t}
\end{equation}

can be written as

\begin{equation} 
    r_j = s_j \cdot (Q_{t_2} - Q_{t_1})
\end{equation}

As $Q_t$ is monotonic, we can simply track the current (latest)
value of $Q$ as a running sum, and snapshot it whenever we expect it to be required for a later computation, i.e. whenever a participant changes their stake.

To compute the total (accumulated) reward for participant $j$, we can then use following formula:

\begin{equation} 
    r_j = s_j \cdot (Q - Q_{t_1})
\end{equation}

\bigskip

\section{Corrected Stake Approach}
In Liquity, collateral and debt shares from liquidations are proportional to the entire collateral of the recipient troves. The redistribution must cope with the fact that some fraction of a trove's entire collateral is the accumulated reward from prior liquidations, and this fraction varies across troves. \\

The "strawman" approach in 6, with rewards proportional to the initial stake, neglects this: it over-rewards fresh troves, and under-rewards older troves. \\

We introduce a corrected stake to restore proportional reward distribution. \\

\begin{equation} \label{eq:2}
    s_i=
        \begin{cases} 
            c_i & for \;totalCollateral_\emptyset = 0\\
            \mathlarger{c_i} \cdot \frac{totalStakes_\emptyset} {\;totalCollateral_\emptyset} & for \;totalCollateral_\emptyset>0
        \end{cases}
\end{equation}

\bigskip
Where $totalStakes_\emptyset$ and $totalCollateral_\emptyset$ are the respective snapshots of the total stakes and total collateral in the system, taken immediately after the last liquidation event. Note that with this terminology, the total collateral includes the total stakes, and therefore $totalCollateral_\emptyset \ge totalStakes_\emptyset$
\\

By extending the original formula from $\cite{Batog}$, we can thus express the collateral and debt share received by a trove by:

\begin{equation}
    r_i=
        \begin{cases} 
            c_j \cdot (Q - Q_{t_1}) & for \;totalCollateral_\emptyset = 0\\
            c_i \cdot \frac{totalStakes_\emptyset} {\;totalCollateral_\emptyset} \cdot (Q - Q_{t_1}) & for \;totalCollateral_\emptyset>0
        \end{cases}
\end{equation}

Note that $r_i$ can stand for the collateral or the debt share of a recipient trove since both of them can be represented as "rewards" through $R_t$.

\section{Outline of the Proof}
As mentioned above, we aim to prove that the corrected stake approach leaves the ordering of the troves by collateral ratio unchanged throughout all liquidations, regardless of any borrowers that change their troves' collateral between liquidation events.

\subsection*{Notation}
Let $ICR$ denote the ratio of a trove's total collateral (valued in USD) to its total debt (in LUSD), including all its previous accumulated rewards from past liquidations. Further, let $ICR_1$ denote the $ICR$ of a trove 1 and $ICR_2$ denote the $ICR$ of a trove 2. \\

We introduce the notion of \textit{system order}. In general, a system of troves increases from order $N$ to order $N+1$ when the following sequence of events occurs:

\begin{itemize}
  \item 1 or more new troves are created 
  \item 1 or more troves are subsequently liquidated
\end{itemize}

\subsection*{Core Proof}
\begin{itemize}
  \item Proof: $ICR_1=ICR_2$. $1^{st}$ order, M past liquidations. Evolves to $2^{nd}$ order: 1 new stake, 1 subsequent liquidation
\end{itemize}


\subsection*{Extensions}
\begin{itemize}
  \item Proof: $ICR_1=ICR_2$ for $1^{st}$ order, M past liquidations Evolves to $2^{nd}$ order: 1 new stake, P subsequent liquidations
  \item Proof: $ICR_1=ICR_2$ for $1^{st}$ order, M past liquidations Evolves to $2^{nd}$ order: Q new stakes, P subsequent liquidations
  \item Show $2^{nd}$ order system is equivalent to first order 
  \item Show that $n^{th}$ order system is equivalent to first order
\end{itemize}


\section{System Order Evolution}
We capture the system order in a system evolution function:

\begin{equation} 
    f(S_N)=S_{N+1}
\end{equation}

\bigskip
Let $S_1$ define a system of troves with past liquidations, in which all active troves have received reward shares from all past liquidations. $S_1$ is a first-order system, and contains only first-order stakes. Each stake $s_i$ is equal to it’s collateral $c_i$, and \textit{totalStakes} $= \sum s_i = \sum c_i$.\\

Let $S_2$ define an evolution of $S_1$, i.e. $S_2 = f(S_1)$. $S_2$ is a system with past liquidations, with \textit{totalStakes} $= \sum s_i + \sum s_j$, where $s_j$ is the stake of newly added $trove_j$. $S_2$ is a second-order system, containing \textbf{both} \textit{first-order} stakes $s_i = c_i$ which have experienced all liquidations, \textbf{and} \textit{second-order} stakes $s_j$ which have only experienced the liquidations after their creation.\\

In general, $S_n$ is a system with n sequential pairs, each consisting of a trove creation period and a liquidation period. trove's made in a given trove creation period t have experienced only those liquidations that occurred in liquidation period t or greater.


\subsection{At First-Order, Stake Equals Initial Collateral}

For first-order systems, all troves were added before any liquidation events occurred. The snapshot $totalCollateral_\emptyset$ is equal to 0. Therefore:

\begin{equation} 
    s_i=c_i
\end{equation}
for all $s_i$, $c_i$ in an $S_1$ system.

\subsection{Intuition Behind Choice of Corrected Stake}

The corrected stake $s_i$ is chosen such that it earns rewards from liquidations equivalent to a trove that would have accumulated $c_i$ total collateral by the time the fresh $trove_i$ was created.\\

The corrected stake effectively models the fresh trove’s collateral $c_i$ as a total collateral, which includes ‘virtual’ accumulated rewards. The corrected stake earns rewards for the trove as if the trove was first-order, and had been in the system from the beginning - thus maintaining proportional reward growth.\\

We now prove that ICR equality is maintained with rewards proportional to corrected stakes - starting with the simplest case, and progressively generalizing.

\section{Corrected Stake Preserves Equality}

\subsection{PROOF. Corrected Stake Preserves ICR Equality Across a Reward Event in a Second Order System with $m$ Past Liquidations}

We consider the following event sequence:

\begin{itemize}
  \item $n+m$ troves are created
  \item $m$ troves are liquidated
  \item A fresh trove is created
  \item An old trove is liquidated
\end{itemize}

\bigskip
In other words, a first-order system of $n+m$ troves undergoes $m$ trove liquidations, before evolving to second-order.\\

Consider the $m$ past liquidations from the point of view of an active first-order $trove_i$. As per (\ref{eq:2}), the stake of $trove_i$ is $s_i = c_i$.\\

Let’s define the reward from liquidating $trove_k$ received by $trove_i$ as the pair $R_i^k = (x_i^k, y_i^k)$ where $x_i^k$ is the share of $trove_k$’s collateral received by $trove_i$ and $y_i^k$ is the share $trove_k$’s debt absorbed by $trove_i$.\\

Let’s call $x_i$ the total accumulated collateral from rewards earned by $trove_i$. At the end of the first-order stage this is the sum of its collateral rewards from $m$ past liquidations.\\

(For simplicity, let’s assume that troves $n+1, ..., n+m$ are the liquidated troves, in that order, and $1, ..., n$ are those that remain)\\

For a remaining trove:\\

\begin{equation} \label{eq:collreward}
  x_i = x_i^{n+1} + ... + x_i^{n+m} = \sum_{k=n+1}^{n+m}x_i^k = \sum_{j=1}^{m}x_i^{n+j}
\end{equation}

For a liquidated trove the sum would be truncated at the previous liquidated trove: \\

\begin{equation}
  x_i = x_i^{n+1} + ... + x_i^{i-1} = \sum_{k=n+1}^{i-1}x_i^k
\end{equation}

With each liquidation, $c_j$ collateral is removed from the system. Again, as per (\ref{eq:2}), stake equals collateral. Thus, the $totalStakes$ numerator in each liquidation is reduced by $c_j$, where $j$ denotes the index of the liquidated trove.\\ 


Let

\begin{equation} 
    C_{n+m}=\sum\limits^{n+m}_{i=1}c_i
\end{equation}

\bigskip
and

\begin{equation} 
    L_m=\sum\limits^m_{j=1}c_{n+j}
\end{equation}

\bigskip
We now sum all reward events, noting that the liquidated trove’s collateral is removed from the $totalStakes$ denominator at each reward. Getting back to (\ref{eq:collreward}), note that for each previous liquidated trove $n+j$, the share of collateral that corresponds to trove $i$ is: \\

\begin{equation} 
  x_i^j = \frac{c_i}{C_{n+m}-L_j}(c_{n+j}+x_{n+j})
\end{equation}

The denominator $C_{n+m}-L_j$ corresponds to the fact that the amount of total initial stakes has been reduced by the amount of liquidated collateral from  troves until $j$, $L_j$. The collateral to redistribute is $c_{n+j}+x_{n+j}$, which corresponds to the initial collateral of the liquidated trove, $c_{n+j}$, plus the collateral it has earned from previous liquidations, $x_{n+j}$. \\

Therefore:\\

\begin{equation} 
    x_i=c_i\left[\frac{c_{n+1}+x_{n+1}}{C_{n+m}-L_1}+\frac{c_{n+2}+x_{n+2}}{C_{n+m}-L_2}+\frac{c_{n+3}+x_{n+3}}{C_{n+m}-L_3}+...+\frac{c_{n+m}+x_{n+m}}{C_{n+m}-L_m}\right]
\end{equation}

\bigskip
i.e.

\begin{equation} 
    x_i=c_i\sum\limits^m_{j=1}\frac{c_{n+j}+x_{n+j}}{\sum\limits^{n+m}_{i=1}c_i-\sum\limits^j_{p=1}c_{n+p}}
\end{equation}

\bigskip
(Note, that for liquidation of a given $trove_j$, the redistributed collateral is the sum of its collateral $c_{n+j}$ plus it’s accumulated collateral reward $x_{n+j}$ which has itself been earned from liquidations $[n+1, n+2, n+3, … n+j-1]$.  Thus, liquidations have a “roll-up” effect - though, it is not important for our result. In fact, it can also be proved that $x_i=c_i\frac{L_m}{C_n}$)\\

We label the main sum expression $H$.\\

Rewriting $trove_i$’s accumulated reward:

\begin{equation} \label{eq:45}
    x_i=Hc_i
\end{equation}

\bigskip
Summing over all $n$ active troves gives the total accumulated rewards for active troves in the system:

\begin{equation} 
    X_n=\sum\limits^n_{i=1}Hc_i
\end{equation}

\begin{equation} \label{eq:47}
    X_n=H \; C_n
\end{equation}

\bigskip
Note that after $m$ liquidations, the system snapshots update from initially:

\begin{equation}
    totalStakes_\emptyset = totalCollateral_\emptyset = C_{n+m}
\end{equation}

to:

\begin{equation} \label{eq:8}
    totalStakes_\emptyset=C_n
\end{equation}

\begin{equation} \label{eq:9}
    totalCollateral_\emptyset=C_n+X_n
\end{equation}

\bigskip
(Note that it can also be proved that that $X_n=L_m$ and therefore $totalCollateral_\emptyset=C_n+L_m=C_{n+m}$)

\bigskip
Now, a fresh trove is added, $trove_F$, with collateral $c_F$.  Let the ICR of $trove_F$ equal the ICR of an active first-order $trove_G$.\\

Now, $trove_Z$ liquidates. Upon liquidation, the second-order $trove_F$ and the first-order $trove_G$ earn the following collateral rewards:

\begin{equation} \label{eq:10}
    ICR_F=ICR_G
\end{equation}

\begin{equation} 
    ICR_F=\frac{c_\mathsmaller{F}}{d_\mathsmaller{F}}
\end{equation}

\begin{equation} 
    ICR_G=\frac{c_\mathsmaller{G}+x_\mathsmaller{G}}{d_\mathsmaller{G}+y_\mathsmaller{G}}
\end{equation}

\bigskip
Where $c_F$, $d_F$ and $c_G$, $d_G$ are the collateral and debt values of $trove_F$ and $trove_G$ respectively.\\

$x_G$, $y_G$ are the respective accumulated collateral and debt rewards for $trove_G$ earned by its stake over its lifetime.\\

The ICR equality identity (\ref{eq:10}) yields the following relation:

\begin{equation} 
        c_F=\frac{d_\mathsmaller{F}}{d_\mathsmaller{G}+y_\mathsmaller{G}}(c_\mathsmaller{G}+x_\mathsmaller{G})
\end{equation}

\bigskip
i.e.

\begin{equation} \label{eq:14}
    c_F=k(c_\mathsmaller{G}+x_\mathsmaller{G})
\end{equation}

\bigskip
where

\begin{equation} \label{eq:15}
    k=\frac{d_\mathsmaller{F}}{d_\mathsmaller{G}+y_\mathsmaller{G}}
\end{equation}


\bigskip
$trove_F$’s stake $s_F$ is given by the corrected stake rule (\ref{eq:2}), that is:

\begin{equation} 
    s_F=\frac{c_\mathsmaller{F} \cdot totalStakes_\emptyset}{totalColateral_\emptyset}
\end{equation}

\bigskip
Which by (\ref{eq:8}) and (\ref{eq:9}) gives:

\begin{equation} \label{eq:17}
    s_F=\frac{c_F \cdot C_n}{C_n+X_n}
\end{equation}

\bigskip
Now, a new liquidation occurs: $trove_Z$ liquidates. The system becomes second-order.\\

The event causes $trove_Z$’s collateral and debt ($c_Z$ and $d_Z$) to be redistributed between all active troves, proportional to their stake.\\

For simplicity, let :

\begin{equation} 
    a=\frac{c_Z+x_Z}{totalStakes}
\end{equation}

\begin{equation} 
    b=\frac{d_Z+y_Z}{totalStakes}
\end{equation}

\bigskip
We define the collateral and debt rewards earned by $trove_F$ and $trove_G$ in the reward event:

\begin{equation} \label{eq:20}
    \begin{split}
        r_{c\mathsmaller{F}}=as_\mathsmaller{F}\\
        r_{d\mathsmaller{F}}=bs_\mathsmaller{F}\\
        r_{c\mathsmaller{G}}=as_\mathsmaller{G}\\
        r_{d\mathsmaller{G}}=bs_\mathsmaller{G}
    \end{split}
\end{equation}

\bigskip
where

\begin{equation} 
    a=\frac{c_Z+x_Z}{totalStakes}
\end{equation}

\begin{equation} 
    b=\frac{d_Z+y_Z}{totalStakes}
\end{equation}

\bigskip
And since $s_\mathsmaller{G}$ is a first-order stake:

\begin{equation} \label{eq:21}
    s_\mathsmaller{G}=c_\mathsmaller{G}
\end{equation}

\bigskip
To show ICR equivalence after the reward event, we must first obtain $s_F$ as a linear function of $c_G$. Recall our definition of $trove_F$’s stake from (\ref{eq:17}):

\begin{equation} 
    s_\mathsmaller{F}=\frac{c_\mathsmaller{F} \cdot C_n}{(C_n+X_n)}
\end{equation}

\bigskip
Now, substituting in the expression for F’s collateral, (\ref{eq:14}), we obtain:

\begin{equation} 
    s_\mathsmaller{F}=\frac{k(c_\mathsmaller{G}+x_\mathsmaller{G})C_n}{C_n+X_n}
\end{equation}


\bigskip
Substituting in the expressions for accumulated reward $x_i$ from (\ref{eq:45}), and total accumulated reward $X_n$ from (\ref{eq:47}):

\begin{equation} 
    s_\mathsmaller{F}=\frac{k(c_G+Hc_G)C_n}{C_n+HC_n}
\end{equation}

\bigskip
And factorizing:

\begin{equation} 
    s_F=\frac{kc_G(C_n+HC_n)}{(C_n+HC_n)}
\end{equation}

\bigskip
Canceling yields:

\begin{equation} \label{eq:29}
    s_F=kc_G
\end{equation}

\bigskip
We now compare ICRs of $trove_F$ and $trove_G$, after liquidation of $trove_Z$.

\begin{equation} 
    ICR_{F \; After}=\frac{c_\mathsmaller{F}+r_{c\mathsmaller{F}}}{d_\mathsmaller{F}+r_{d\mathsmaller{F}}}
\end{equation}

\begin{equation} 
    ICR_{G \; After}=\frac{c_\mathsmaller{G}+x_\mathsmaller{G}+r_{c\mathsmaller{G}}}{d_\mathsmaller{G}+y_\mathsmaller{G}+r_{d\mathsmaller{G}}}
\end{equation}

\bigskip
Using (\ref{eq:20}), the individual rewards as functions of stakes:

\begin{equation} 
    ICR_{F \; After}=\frac{c_F+as_F}{d_F+bs_F}
\end{equation}

\begin{equation} 
    ICR_{G \; After}=\frac{c_\mathsmaller{G}+x_\mathsmaller{G}+as_\mathsmaller{G}}{d_\mathsmaller{G}+y_\mathsmaller{G}+bs_\mathsmaller{G}}
\end{equation}

\bigskip
Now, substituting our definitions for $s_\mathsmaller{G}$ (\ref{eq:21}) and $s_F$ (\ref{eq:29}):

\begin{equation} 
    ICR_{F \; After}=\frac{c_\mathsmaller{F}+akc_\mathsmaller{G}}{d_\mathsmaller{F}+bkc_\mathsmaller{G}}
\end{equation}

\begin{equation} 
    ICR_{G \; After}=\frac{c_\mathsmaller{G}+x_\mathsmaller{G}+ac_\mathsmaller{G}}{d_\mathsmaller{G}+y_\mathsmaller{G}+bc_\mathsmaller{G}}
\end{equation}

\bigskip
Using identities (\ref{eq:14}) for $c_F$, and (\ref{eq:15}) for $d_F$:

\begin{equation} 
    ICR_{F \; After}=\frac{k(c_\mathsmaller{G} + x_\mathsmaller{G}+ac_\mathsmaller{G})}{k(d_\mathsmaller{G}+y_\mathsmaller{G}+bc_\mathsmaller{G})}
\end{equation}

\begin{equation} 
    ICR_{G \; After}=\frac{c_\mathsmaller{G}+x_\mathsmaller{G}+ac_\mathsmaller{G}}{d_\mathsmaller{G}+y_\mathsmaller{G}+bc_\mathsmaller{G}}
\end{equation}

\bigskip
Cancelling $k$:

\begin{equation} 
        ICR_{F \; After}=\frac{c_\mathsmaller{G} + x_\mathsmaller{G}+ac_\mathsmaller{G}}{d_\mathsmaller{G}+y_\mathsmaller{G}+bc_\mathsmaller{G}}
\end{equation}

\begin{equation} 
    ICR_{G \; After}=\frac{c_\mathsmaller{G}+x_\mathsmaller{G}+ac_\mathsmaller{G}}{d_\mathsmaller{G}+y_\mathsmaller{G}+bc_\mathsmaller{G}}
\end{equation}

\bigskip
Thus:

\begin{equation} 
    ICR_{F \; After}=ICR_{G \; After}
\end{equation}

\bigskip
QED.

\subsection{EXTENSION PROOF. Arbitrary Number of Liquidation Events At Current System Order}

If instead of a single liquidation event at a given system order, we have $P$ liquidation events, it is clear that ICR equality holds across all $P$ events:\\

Since ICR equality holds across one liquidation event, it will hold across the next, and thus hold for all.\\

Liquidation events do not alter the stakes that earn shares of liquidated collateral and debt - and for a given stake, the individual trove reward term $x_i$ given in 4) depends only on reward sizes and the total stakes.

\subsection{EXTENSION PROOF. Arbitrary Number of troves Added Between Liquidation Events}

With $N$ second-order troves added between consecutive liquidation events, the stake $s_F$ of any given second-order trove is given by 1): 

\begin{equation} 
    s_F=\frac{c_F \cdot totalStakes_\emptyset}{totalColateral_\emptyset}
\end{equation}

\bigskip
The snapshots of the system state after the last liquidation event ($totalStakes_\emptyset$, $totalColateral_\emptyset$) remain constant until the next liquidation. It is clear that all $N$ second-order stakes $s_F$ have been corrected by the same constant factor.\\

Thus, $s_F$ in the N second-order troves case is equal to $s_F$ in the single second-order trove case.\\

As such, the logic of the Main Proof applies - and ICR equality between a second-order trove and first-order trove holds across a liquidation event, no matter how many fresh troves are added in between.

\subsection{CONCLUSION 1}

\bigskip
Combining Main Proof with Extensions 1 \& 2 yields the following conclusion:\\

\uline{In a second order system with $M$ previous liquidations, and $N$ second-order troves added after the last liquidation, ICR equality between a first-order trove and second-order trove holds across $P$ subsequent liquidation events.}\\

\subsection{2nd Order Systems Collapse to 1st Order}
We now show that a second-order system is equivalent to a first-order system.\\

Consider a hypothetical first order $trove_1$ and an actual second order $trove_2$. Let both troves have identical ICR, and also let $trove_1$’s total collateral and debt equal $trove_2$’s initial collateral and initial debt respectively:

\begin{equation} 
    c_1+x_1=c_2
\end{equation}

\begin{equation} 
    d_1+y_1=d_2
\end{equation}

\bigskip
Clearly, the ratio  $k = \frac{d_2}{(d_1+y_1)} = 1$.\\

We substitute $k=1$ into the second-order system expression for $s_F$, from equation 55), to yield:

\begin{equation} 
    s_2=c_1
\end{equation}

\bigskip
Thus, any second-order stake is equivalent to some hypothetical first-order stake $s_1=c_1$, which has accumulated collateral reward $x_1=(c_2-c_1)$ and debt reward $y_1=(d_2-d_1)$.\\

Therefore any second order system is equivalent to a first order system that contains only first-order stakes which have experienced all liquidations. We write:

\begin{equation} 
    S_2=S_1
\end{equation}

\subsection{N’th Order Systems Collapse to 1st Order}
We prove it by induction. We have already proved for $n=1$ that $S_1=S_2$.
Now we show that if it’s true for $n-1$ then it’s true for $n$, i.e.:

\begin{equation}
    S_{n-1} = S_n \Rightarrow S_n = S_{n+1}
\end{equation}

Recall our system evolution function: 

\begin{equation} 
    f(S_N)=S_{N+1}
\end{equation}

Therefore:

\begin{equation} 
    S_{N+1} = f(S_N) = f(S_{n-1}) = S_N
\end{equation}

\bigskip
So, for every $N$, $S_N = S_{N-1}$, and for the transitive property of equivalence, we finally have:

\begin{equation}
    S_N=S_1
\end{equation}

\bigskip
Having shown all nth order systems are equivalent to a first order system, we now extend our previous conclusion to nth order systems:

\subsection{CONCLUSION 2}

\uline{In an $n^{th}$ order system with $M$ previous liquidations, and $N$ second-order troves added after the last liquidation, ICR equality between an $(n-1)^{th}$ order trove and $n^{th}$ order trove holds across $P$ liquidation events.}

\section{Corrected Stake Preserves Order}
Here we show that ICR ordering is preserved with corrected stakes across a liquidation event.\\

We make use of the first-order equivalence result, namely, that with corrected stakes:

\begin{equation} 
    S_N = S_1
\end{equation}

i.e:\\

Any N’th order system of troves is equivalent to a first-order system of troves. For a given fresh trove with stake $s_i$ and collateral $c_i$, the stake $s_i$ is equivalent to some hypothetical first-order stake $c_j$ which has accumulated collateral reward $x_j = (c_i - c_j)$ and debt reward $y_j = (d_i - d_j)$.\\

Due to this equivalence between first and N’th-order systems, if ordering is preserved for first-order systems, it is preserved for N’th order systems.\\

Now consider a first-order system of troves, with stakes equal to their initial collateral.\\

Let $trove_1$ and $trove_2$ be troves with initial collateral $c_1$, $c_2$ accumulated collateral and debt rewards $x_1$, $y_1$ and $x_2$, $y_2$ respectively:\\

\begin{equation} 
    ICR_1=\frac{c_1+x_1}{d_1+x_1}
\end{equation}

\begin{equation} 
    ICR_2=\frac{c_2+x_2}{d_2+y_2}
\end{equation}

Let their ICRs be such that:

\begin{equation} 
    ICR_1 > ICR_2
\end{equation}

\bigskip
Since, a first-order trove’s collateral and debt rewards are always in direct proportion to its initial collateral, we can write the accumulated rewards as:

\begin{equation} 
    x_1=Ac_1
\end{equation}

\begin{equation} 
    x_2=Ac_2
\end{equation}

and

\begin{equation} 
    y_1=Bc_1
\end{equation}

\begin{equation} 
    y_2=Bc_2
\end{equation}

\bigskip
Where A is the sum of all ‘collateral rewards per unit staked’, and B is the sum of all ‘debt rewards per unit staked’. This yields ICRs:

\begin{equation} 
    ICR_1=c_1\frac{1+A}{d_1+Bc_1}
\end{equation}

\begin{equation} 
    ICR_2=c_2\frac{1+A}{d_2+Bc_2}
\end{equation}

And the initial ICR inequality becomes:

\begin{equation} 
    c_1\frac{1+A}{d_1+Bc_1}>c_2\frac{1+A}{d_2+Bc_2}
\end{equation}

\bigskip
Cross multiplying and cancelling the common denominator yields:

\begin{equation} 
    c_1\left(1+A\right)\left(d_2+Bc_2\right)>c_2\left(1+A\right)\left(d_1+Bc_1\right)
\end{equation}

Then expanding:

\begin{equation} 
    c_1\left(d_2+Bc_2\right)>c_2\left(d_1+Bc_1\right)
\end{equation}

\begin{equation} 
    c_1d_2+Bc_1c_2 > c_2d_1+Bc_1c_2
\end{equation}

\bigskip
And cancelling terms:

\begin{equation}
    c_1d_2 > c_2d_1
\end{equation}

\bigskip
Finally yielding the result:

\begin{equation} \label{eq:217}
    \frac{d_2}{c_2}>\frac{d_1}{c_1}
\end{equation}

\bigskip
We will later use this to prove that the inequality of ICRs holds across a liquidation event.\\

Now consider a liquidation event occurs. Upon a trove liquidation, $r_c$ collateral and $r_d$ debt are distributed to all active troves. Each active trove earns rewards proportional to its initial collateral, thus:

\begin{equation} 
    ICR_{1After}=\frac{c_1\left(1+A\right)+ac_1}{d_1+Bc_1+bc_1}
\end{equation}

\begin{equation} 
    ICR_{2After}=\frac{c_2\left(1+A\right)+ac_2}{d_2+Bc_2+bc_2}
\end{equation}

\bigskip
Where:

\begin{equation} 
    a=\frac{r_c}{totalStakes}
\end{equation}

\begin{equation} 
    b=\frac{r_d}{totalStakes}
\end{equation}

\bigskip
Collecting terms:

\begin{equation} 
    ICR_1=\frac{c_1\left(1+a+A\right)}{d_1+\left(1+B\right)c_1}
\end{equation}

\begin{equation} 
    ICR_2=\frac{c_2\left(1+a+A\right)}{d_2+\left(1+B\right)c_2}
\end{equation}

\bigskip
And taking reciprocals:

\begin{equation} 
    \frac{1}{ICR_{1After}}=\frac{d_1+\left(1+B\right)c_1}{c_1\left(1+a+A\right)}
\end{equation}

\begin{equation} 
    \frac{1}{ICR_{2After}}=\frac{d_2+\left(1+B\right)c_2}{c_2\left(1+a+A\right)}
\end{equation}

\bigskip
Rearranging, and separating the constant term:

\begin{equation} 
    \frac{1}{ICR_{1After}}=\left[\frac{\frac{d_1}{c_1}}{1+a+A}\right]+\left[\frac{1+B}{1+a+A}\right]
\end{equation}

\begin{equation} 
    \frac{1}{ICR_{2After}}=\left[\frac{\frac{d_2}{c_2}}{1+a+A}\right]+\left[\frac{1+B}{1+a+A}\right]
\end{equation}

\bigskip
Recall our earlier result (\ref{eq:217}): $\frac{d_1}{c_1}<\frac{d_2}{c_2}$. Thus:

\begin{equation} 
    \frac{1}{ICR_{1After}}<\frac{1}{ICR_{2After}}
\end{equation}

\bigskip
Then taking reciprocals, finally yields:

\begin{equation} 
    ICR_{1After}>ICR_{2After}
\end{equation}

\bigskip
Therefore, trove ordering holds across a liquidation event in first-order systems, and thus holds across a liquidation event in N’th order systems.

\bigskip
\subsection*{References}

\begin{thebibliography}{9}

\bibitem{Whitepaper} 
R. Lauko, R. Pardoe. 
\textit{Liquity: Decentralized Borrowing Protocol (Whitepaper), 2020}. 
https://docsend.com/view/bwiczmy

\bibitem{Batog} 
B. Batog, L. Boca, N. Johnson.
\textit{Scalable Reward Distribution on the Ethereum Blockchain, 2018}. 
http://batog.info/papers/scalable-reward-distribution.pdf


\end{thebibliography}

\end{document}
